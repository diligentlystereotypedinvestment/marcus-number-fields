\documentclass[a4paper]{article}

\input{~/templates/math.tex}

\title{Selected Exercises from Marcus' Number Fields}
\author{Vincent Tran}

\begin{document}
\maketitle

\section{Chapter 1}

For questions 30-32: $R $ is an integral domain (commutative ring with 1 and no zero divisors)

30. Show that two ideals in $R $ are isomorphic as $R $-modules iff they are in the same ideal class.
\begin{proof}
	If two ideals $I,J $ are in the same ideal class s.t. $\alpha I = \beta J $, map an element $i\in J $ to $\beta ^{-1} \alpha i $ $\beta ^{-1}\alpha  $ exists because $\alpha I = \beta J $.
	This is an isomorphism since $\alpha I = \beta J $ and it respects the additive structure ($i_{1}+i_{2} \mapsto \beta ^{-1} \alpha (i_{1}+i_{2}) = \beta ^{-1}\alpha (i_{1})+\beta ^{-1}\alpha i_{2}$).

	For the reverse direction:
	Given isomorphic $R $ modules $I,J $, we can
\end{proof}

\section{Chapter 1}

26.

\begin{proof}
	We do as in Theorem 11:

	Because the $\beta_i $'s generate the same additive subgroup as $\gamma_i $'s, we can write each $\beta _i $ as a sum of $\gamma_i $'s.
	Hence we have a matrix $M $ with coefficients in $\Z $ s.t.
	\[
		\begin{pmatrix} \beta _1\\ \vdots\\ \beta _n \end{pmatrix} = M \begin{pmatrix} \gamma _1\\ \vdots\\ \gamma _n \end{pmatrix}
	.\]
	As the coefficients of $M $ are fixed by $\sigma_i $, we get that
	\[
		[\sigma_j(\beta_i)] = M[\sigma_j(\gamma_i)]
	.\]
	Hence
	\[
		\text{disc}(\beta_{1},\ldots ,\beta _n) = |M|^2\text{disc}(\gamma_{1},\ldots ,\gamma _n)
	.\]
	As $M $ has coefficients in $\Z $, $|M| \in \Z \implies \text{disc}(\gamma_{1},\ldots,\gamma _n) \mid \text{disc}(\beta_{1},\ldots ,\beta_n)$.
	Similarly, $\text{disc}(\beta_{1},\ldots ,\beta _n) | \text{disc}(\gamma_{1},\ldots ,\gamma_n) $.

	Therefore they are equal.
\end{proof}

27a.

\begin{proof}
	Write $G = \Z \oplus \Z \oplus \cdots \oplus \Z $.
	Then $H $, as a subset of $G $, is a subset in each coordinate.
	Thus $H = m_{1}\Z \oplus m_{2}\Z \oplus \cdots \oplus m_n\Z$.
	Hence $G / H = \Z / m_{1} \Z \oplus \Z / m_{2}\Z \oplus \cdots \oplus \Z / m_n\Z$.

	This is clearly a finite group.
\end{proof}

27b.

\begin{proof}
	The existence of a generating set is obvious (just take $(1,\ldots,0),(0,1,\ldots ,0),\ldots ,(0,\ldots ,1) $).
\end{proof}

27c.

\begin{proof}

\end{proof}

27d.

\begin{proof}
	Using 27c:
	Let $H = \alpha_{1}\Z \oplus \cdots \oplus \alpha_n\Z $.

	If they form an integral basis, then $\disc H  = |\{e\} |^2\disc R  $ ($R / H = \{e\}  $ since $R = H$).

	If $\disc H = \disc R $, then by 27c, $\disc H = |R / H|^2\disc R = \disc R \iff |R / H|^2 = 1$.
	Thus $R / H = \{e\}   $ and $H = R $, showing that $\alpha_{1},\ldots ,\alpha _n $ integral basis.

	Using matrices:

	We have
	\[
		A \coloneqq \begin{pmatrix} \alpha_1\\ \vdots\\ \alpha_n \end{pmatrix} = M \begin{pmatrix} x_{1}\\ \vdots \\ x_n \end{pmatrix}
	\]
	where $x_i $ is an integral basis.
	Also define $X = \begin{pmatrix} x_{1}\\ \vdots \\ x_n \end{pmatrix}  $.
	Hence
	\[
		\begin{pmatrix} \sigma_1\ A & \sigma_2\ A & \cdots & \sigma_n\ A \end{pmatrix} = \begin{pmatrix} \sigma_1 MX & \cdots & \sigma_n MX \end{pmatrix} = \begin{pmatrix} M\sigma_1 X & \cdots & M\sigma_n X \end{pmatrix} = M\begin{pmatrix} \sigma_1 X & \cdots & \sigma_n X \end{pmatrix}
	\]
	as $M $ is over $\Z $ and facts from linear algebra.
	Therefore
	\[
		\disc\ H = \left|\begin{pmatrix} \sigma_1\ A & \sigma_2\ A & \cdots & \sigma_n\ A \end{pmatrix}\right|^2 = \left|M\begin{pmatrix} \sigma_1 X & \cdots & \sigma_n X \end{pmatrix}\right|^2 = |M|^2 \disc\ G
	.\]

	So $\disc\ G = \disc\ H $ iff $|M|^2 =1$ iff $M $ is invertible.
	$M $ is invertible iff $\alpha _1,\ldots ,\alpha_n$ is an basis as well (any linear relation using $x_i $'s can be converted into an expression using $\alpha_i $'s and likewise).
\end{proof}

27e.

\begin{proof}
	Let $H = \alpha_{1}\Z \oplus \cdots \oplus \alpha_n\Z $.
	Since $\disc H = |G / H|^2 \disc G$, $\disc H $ is only square-free if $|G / H| $ is one.
	Thus $G = H $ and $\alpha_{1},\ldots ,\alpha_n $ form an integral basis for $G $.
\end{proof}

\end{document}
